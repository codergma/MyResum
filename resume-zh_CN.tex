% !TEX TS-program = xelatex
% !TEX encoding = UTF-8 Unicode
% !Mode:: "TeX:UTF-8"

\documentclass{resume}
\usepackage{zh_CN-Adobefonts_external} % Simplified Chinese Support using external fonts (./fonts/zh_CN-Adobe/)
%\usepackage{zh_CN-Adobefonts_internal} % Simplified Chinese Support using system fonts
\usepackage{linespacing_fix} % disable extra space before next section
\usepackage{cite}

\begin{document}
\pagenumbering{gobble} % suppress displaying page number

\name{刘彬}

% {E-mail}{mobilephone}{homepage}
% be careful of _ in emaill address
\contactInfo{codergma@163.com}{132-0712-2236}{http://blog.codergma.com}
% {E-mail}{mobilephone}
% keep the last empty braces!
%\contactInfo{xxx@yuanbin.me}{(+86) 131-221-87xxx}{}

\section{\faInfo\ 求职意向}
% increase linespacing [parsep=0.5ex]
\begin{itemize}[parsep=0.5ex]
  \item php服务端开发,服务器架构
  \item 不抵制前端编程,也不擅长
\end{itemize}
 
\section{\faGraduationCap\  教育背景}
\datedsubsection{\textbf{中国地质大学}}{2010.09 -- 2014.07}


\section{\faUsers\ 工作经历}
\datedsubsection{\textbf{中地数码} 武汉}{2014年7月 -- 至今}
PHP后端开发
\begin{itemize}
  \item 文件服务器开发
  \item 用户中心开发
  \item 需求中心开发
\end{itemize}
阿里云环境部署及维护
\begin{itemize}
  \item 服务器架构及运维
  \item 服务器优化
\end{itemize}
其他
\begin{itemize}
  \item gitlab代码服务器搭建及维护
  \item 备份服务器搭建及维护
  \item 内网反向代理搭建及维护
  \item 内网dns搭建及维护
\end{itemize}

% Reference Test
%\datedsubsection{\textbf{Paper Title\cite{zaharia2012resilient}}}{May. 2015}
%An xxx optimized for xxx\cite{verma2015large}
%\begin{itemize}
%  \item main contribution
%\end{itemize}

\section{\faCogs\ IT 技能}
% increase linespacing [parsep=0.5ex]
\begin{itemize}[parsep=0.5ex]
  \item 编程语言: PHP, html, css, js
  \item 编程框架: CI, ThinkPhp, Jquery, Bootstrap  
  \item 平台: Ubuntu
  \item 版本控制器: git
\end{itemize}

\section{\faHeartO\ 自我描述}
\begin{itemize}
  \item 技术博客: http://blog.codergma.com
  \item GitHub: https://github.com/codergma
  \item 熟悉HTTP,WebSocket,TCP/UDP等协议。
  \item 熟练使用php,了解php内核,了解php进程编程,熟悉CI,TP编程框架。
  \item 熟悉apache配置,以及ngnix负载均衡的配置。
  \item 熟悉mysql应用开发,mysql读写分离,主从服务器搭建,日志分析,性能优化。
  \item 熟练使用Redis,了解基本配置和存储结构优化,smaryun的用户中心,github上项目UcenterRedis就是借助redis实现的。
  \item 写的了爬虫(https://github.com/codergma/FaceBook),爬取FaceBook数据及模拟操作;
  \item 有良好的工作习惯,熟练使用documents,stackoverflow,github,google, 博客等引擎或社区解决问题,不哭爹喊娘;
\end{itemize}



%% Reference
%\newpage
%\bibliographystyle{IEEETran}
%\bibliography{mycite}
\end{document}
